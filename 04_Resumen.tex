% No modificar estas líneas de código, por favor dirigirse a RESUMEN y ABSTRACT
\newpage

%------------------------
%
%       RESUMEN 
%
%------------------------

\section*{RESUMEN}

El acceso limitado a los datos críticos de las autoclaves, como temperatura, presión y ciclos de operación, representa un desafío significativo para la empresa ITM. Este problema dificulta la detección oportuna de fallos, alarga los tiempos de inactividad y aumenta los costos asociados al mantenimiento correctivo. Actualmente, la gestión de estos equipos se realiza de forma manual, lo que limita la capacidad de respuesta ante eventualidades y disminuye la eficiencia operativa.  

El objetivo principal del proyecto es implementar un sistema de monitoreo y operación en tiempo real para las autoclaves de ITM, utilizando tecnología IoT. Este sistema permitirá recopilar y procesar datos de manera continua, proporcionando acceso remoto a información clave sobre el estado de los equipos. De esta forma, se busca transformar la gestión de las autoclaves mediante una plataforma tecnológica que facilite el mantenimiento predictivo.  

El sistema desarrollado integra sensores, microcontroladores como el ESP32 y protocolos de comunicación como MQTT, para transmitir datos a una plataforma web. Esta solución no solo permite monitorizar las condiciones de funcionamiento, sino que también emplea algoritmos de análisis para predecir posibles fallos y programar mantenimientos de manera anticipada. En conclusión, esta innovación tecnológica optimiza la gestión de recursos, minimiza tiempos de inactividad y asegura la operación eficiente de los equipos.

\textbf{Palabras clave:} Autoclave, Monitoreo, Remoto, Tiempo Real, IoT, Mantenimiento, Predictivo, ESP32, MQTT.

\newpage

%------------------------
%
%       ABSTRACT
%
%------------------------

\section*{ABSTRACT}
The limited access to critical autoclave data, such as temperature, pressure, and operation cycles, presents a significant challenge for ITM. This issue hinders the timely detection of failures, prolongs downtime, and increases the costs associated with corrective maintenance. Currently, the management of these devices is performed manually, limiting response capabilities and reducing operational efficiency.

The primary objective of the project is to implement a real-time monitoring and operation system for ITM’s autoclaves using IoT technology. This system will continuously collect and process data, providing remote access to key information about the equipment's status. The aim is to transform autoclave management through a technological platform that enables predictive maintenance.

The developed system integrates sensors, microcontrollers like the ESP32, and communication protocols such as MQTT to transmit data to a web platform. This solution not only allows for the monitoring of operating conditions but also employs analytical algorithms to predict potential failures and schedule maintenance proactively. In conclusion, this technological innovation optimizes resource management, minimizes downtime, and ensures the efficient operation of the equipment.

\textbf{Keywords:} Autoclave, Monitoring, Remote, Real-Time, IoT, Maintenance, Predictive, ESP32, MQTT.