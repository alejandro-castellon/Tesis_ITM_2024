% No modificar estas líneas de código, por favor dirigirse a RECOMENDACIONES

\newpage
\phantomsection\addcontentsline{toc}{section}{RECOMENDACIONES} 

%------------------------
%
%       RECOMENDACIONES
%
%------------------------

\section*{RECOMENDACIONES}
Finalmente como recomendaciones tenemos la extensión del monitoreo a otros equipos, se recomienda ampliar el alcance del sistema de monitoreo remoto a otros equipos críticos utilizados en las instalaciones. Esto permitiría optimizar aún más los procesos operativos y de mantenimiento, aprovechando la infraestructura tecnológica existente para garantizar una gestión centralizada y eficiente de los equipos.

Por otra parte, la mejora continua del análisis predictivo es fundamental para seguir perfeccionando los algoritmos de análisis de datos para mejorar la precisión de las predicciones de fallos. Invertir en herramientas avanzadas de análisis y aprendizaje automático podría incrementar la capacidad del sistema para anticiparse a problemas, asegurando una mayor fiabilidad y disponibilidad de las autoclaves.

Por ultimo es de suma importancia que exista la capacitación y sensibilización de los usuarios,se recomienda realizar programas de capacitación para los operadores y técnicos responsables del sistema, enfocándose en la interpretación de los datos y el uso de la plataforma web. Asimismo, sensibilizar sobre la importancia del mantenimiento predictivo ayudará a consolidar el uso de esta tecnología como una herramienta clave para la eficiencia operativa.
