% No modificar estas líneas de código, por favor dirigirse a CONCLUSIONES

\newpage
\phantomsection\addcontentsline{toc}{section}{CONCLUSIONES} 

%------------------------
%
%       CONCLUSIONES
%
%------------------------

\section*{CONCLUSIONES}

La implementación del sistema de monitoreo remoto de autoclaves a través de una página web ha demostrado ser una solución efectiva para optimizar la operación y el mantenimiento de estos equipos. Gracias al acceso en tiempo real a datos como temperatura, presión y ciclos de operación, los operadores pueden tomar decisiones más informadas, reduciendo significativamente los tiempos de respuesta ante fallos o anomalías. Este enfoque ha permitido mejorar la eficiencia operativa, asegurando que las autoclaves funcionen de manera continua y confiable, lo que impacta positivamente en la productividad y calidad de los procesos.

Respecto al mantenimiento predictivo eficiente, el análisis de los datos recopilados por el sistema ha sido clave para implementar un enfoque de mantenimiento predictivo. Al identificar patrones y tendencias en el comportamiento de las autoclaves, se han podido predecir posibles fallos antes de que ocurrieran, lo que ha permitido programar intervenciones preventivas de manera eficiente. Esto no solo ha reducido los costos asociados al mantenimiento correctivo, sino que también ha extendido la vida útil de los equipos y garantizado su disponibilidad operativa.

Sobre la innovación tecnológica y accesibilidad a datos de autoclaves en \acrshort{itm}, la integración de tecnologías como \acrshort{iot}, ESP32 y el protocolo \acrshort{mqtt} ha permitido desarrollar un sistema robusto y accesible desde cualquier dispositivo con conexión a internet. Este avance no solo representa un hito en la modernización del mantenimiento de autoclaves, sino que también sienta las bases para futuras aplicaciones tecnológicas en otros equipos industriales. La plataforma web no solo facilita el monitoreo remoto, sino que también mejora la experiencia de los usuarios al centralizar toda la información relevante en una interfaz amigable y de fácil acceso.